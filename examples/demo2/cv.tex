%% Tianyuan Liu - Academic CV
%% Compiled with: pdflatex cv.tex

\documentclass[11pt,a4paper,sans]{moderncv}

% moderncv themes
\moderncvstyle{banking}
\moderncvcolor{blue}

% character encoding
\usepackage[utf8]{inputenc}

% adjust page margins
\usepackage[scale=0.85]{geometry}

% bibliography
\usepackage[backend=biber,style=numeric,sorting=ydnt,maxbibnames=99]{biblatex}
\addbibresource{cv_publications.bib}

% personal data
\name{Tianyuan}{Liu}
\title{Computational Genomics Researcher}
\address{Paterna, Valencia}{Spain}
\phone[mobile]{+34~601569170}
\email{tianyuan.liu@csic.es}
\homepage{tianyuan-liu.github.io}
\social[github]{tianyuan-liu}

% reduce spacing in lists
\usepackage{enumitem}
\setlist{noitemsep}

\begin{document}

\makecvtitle

\section{About}
Computational genomics researcher specializing in long-read sequencing technologies and transcriptome analysis. Part of the LongTREC (Long-read Transcriptomics Research and Education Consortium) Marie Skłodowska-Curie Actions program. Research focuses on developing computational methods for analyzing long-read RNA sequencing data, with particular expertise in isoform regulatory biology and multiomics integration.

\section{Education}
\cventry{2023--Present}{PhD Student \& Early Stage Researcher}{Institute for Integrative Systems Biology}{Spanish National Research Council (CSIC-UV)}{Valencia, Spain}{Advisor: Dr. Ana Conesa\\LongTREC Marie Skłodowska-Curie Actions program\\Focus: Long-read multiomics methods for isoform regulatory biology}
\cventry{2016--2020}{Bachelor of Engineering in Bioengineering}{Huazhong Agricultural University (HZAU)}{China}{}{Comprehensive education in bioengineering with focus on computational biology and bioinformatics applications}

\section{Research Focus}
\cvitem{Main Topic}{Long-read multiomics methods to understand isoform regulatory biology}
\cvitem{Areas}{Quality control methods, isoform detection, epigenome profiling, transcriptome assembly, differential analysis}

\section{Research Experience}
\cventry{2023--Present}{PhD Student \& Early Stage Researcher}{Institute for Integrative Systems Biology, CSIC-UV}{Valencia, Spain}{}{Developing computational methods for isoform regulatory biology\\Contributing to major software projects like SQANTI3\\Collaborating with international research networks}
\cventry{2019--2020}{Research Experience}{University of Florida, Genetics Institute}{USA}{}{Supervised by Dr. Ana Conesa and Dr. Guillem Ylla\\Working on MirCure project for microRNA quality control and curation methods}

\section{Publications}

\subsection{Peer-Reviewed Articles}

\begin{enumerate}[leftmargin=*,label=\arabic*.]

\item \textbf{Liu, T.}, \& Conesa, A. (2025). Profiling the epigenome using long-read sequencing. \textit{Nature Genetics}, 57(1), 27--41.

\item Monzó, C., \textbf{Liu, T.}, \& Conesa, A. (2025). Transcriptomics in the era of long-read sequencing. \textit{Nature Reviews Genetics}, 1--21.

\item Pardo-Palacios, F.J., Arzalluz-Luque, A., Kondratova, L., Salguero, P., Mestre-Tomás, J., Amorín, R., Estevan-Morió, E., \textbf{Liu, T.}, et al. (2024). SQANTI3: curation of long-read transcriptomes for accurate identification of known and novel isoforms. \textit{Nature Methods}, 21(5), 793--797.

\item Pardo-Palacios, F.J., Wang, D., Reese, F., Diekhans, M., Carbonell-Sala, S., ..., \textbf{Liu, T.}, ..., Conesa, A. (2024). Systematic assessment of long-read RNA-seq methods for transcript identification and quantification. \textit{Nature Methods}, 21(7), 1349--1363.

\item Mestre-Tomás, J., \textbf{Liu, T.}, Pardo-Palacios, F., \& Conesa, A. (2023). SQANTI-SIM: a simulator of controlled transcript novelty for lrRNA-seq benchmark. \textit{Genome Biology}, 24(1), 286.

\item \textbf{Liu, T.}, Salguero, P., Petek, M., Martínez-Mira, C., Balzano-Nogueira, L., Conesa, A., et al. (2022). PaintOmics 4: New tools for the integrative analysis of multi-omics datasets supported by multiple pathway databases. \textit{Nucleic Acids Research}, 50(W1), W551--W559.

\item \textbf{Liu, T.}, Balzano-Nogueira, L., Lleó, A., \& Conesa, A. (2020). Transcriptional differences for COVID-19 disease map genes between males and females indicate a different basal immunophenotype relevant to the disease. \textit{Genes}, 11(12), 1447.

\item Ylla, G., \textbf{Liu, T.}, \& Conesa, A. (2020). MirCure: a tool for quality control, filter and curation of microRNAs of animals and plants. \textit{Bioinformatics}, 36(Supplement\_2), i618--i624.

\end{enumerate}

\subsection{Preprints \& Under Review}

\begin{enumerate}[leftmargin=*,label=\arabic*.,resume]

\item \textbf{Liu, T.}, Paniagua, A., Jetzinger, F., Ferrández-Peral, L., Frankish, A., \& Conesa, A. (2025). Transcriptome Universal Single-isoform COntrol (TUSCO): A Framework for Evaluating Transcriptome Reconstruction Quality. \textit{bioRxiv}, 2025.08.23.671926. (Under review at \textit{Nature Communications})

\end{enumerate}

\section{Conference Presentations}
\cvitem{2025}{Transcriptome Universal Single-isoform Control (TUSCO): A Framework for Evaluating Transcriptome Quality. \textit{Oral presentation}, ISMB/ECCB 2025, Liverpool, United Kingdom. Track: iRNA: Integrative RNA Biology.}

\section{Major Projects}
\cventry{}{SQANTI3}{}{}{}{Comprehensive quality control and curation pipeline for long-read transcriptomes. Contribution: Report generation and documentation.}
\cventry{}{SQANTI-SIM}{}{}{}{Simulator of controlled transcript novelty for long-read RNA-seq benchmarking. Contribution: Added PBSIM integration to the simulation framework.}
\cventry{}{PaintOmics 4}{}{}{}{Advanced web-based platform for integrative analysis of multi-omics datasets with support for multiple pathway databases. Technologies: JavaScript, web-based interface, multi-omics integration.}
\cventry{}{MirCure}{}{}{}{Specialized tool for microRNA quality control, filtering, and curation in both animals and plants. Language: R, bioinformatics pipeline.}
\cventry{}{DeCovid}{}{}{}{Tool for conducting wide-spread transcriptional analysis for COVID-19 Disease Map genes. Features: Differential expression analysis, GO enrichment, interactive web interface.}

\section{Technical Skills}

\subsection{Programming Languages}
\cvitem{}{Python, R, JavaScript, HTML/CSS, Bash/Shell}

\subsection{Technologies \& Tools}
\cvitem{}{PacBio, Oxford Nanopore, Bioconductor, Docker, Git/GitHub, Nextflow, Conda/Mamba}

\subsection{Data Analysis \& Methods}
\cvitem{}{Statistical analysis, machine learning, data visualization, pipeline development, reproducible research}

\subsection{High-Performance Computing}
\cvitem{}{SLURM, cluster workflows, cloud integration}

\vfill
\begin{center}
\textit{\small Last updated: \today}
\end{center}

\end{document}
